\documentclass[10pt]{article}

% This is a helpful package that puts math inside length specifications
\usepackage{calc}


% Simpler bibsection for CV sections
% (thanks to natbib for inspiration)
\makeatletter
\newlength{\bibhang}
\setlength{\bibhang}{1em}
\newlength{\bibsep}
 {\@listi \global\bibsep\itemsep \global\advance\bibsep by\parsep}
\newenvironment{bibsection}%
        {\vspace{-\baselineskip}\begin{list}{}{%
       \setlength{\leftmargin}{\bibhang}%
       \setlength{\itemindent}{-\leftmargin}%
       \setlength{\itemsep}{\bibsep}%
       \setlength{\parsep}{\z@}%
        \setlength{\partopsep}{0pt}%
        \setlength{\topsep}{0pt}}}
        {\end{list}\vspace{-.6\baselineskip}}
\makeatother

% Layout: Puts the section titles on left side of page
\reversemarginpar
	
%
%         PAPER SIZE, PAGE NUMBER, AND DOCUMENT LAYOUT NOTES:
%
% The next \usepackage line changes the layout for CV style section
% headings as marginal notes. It also sets up the paper size as either
% letter or A4. By default, letter was used. If A4 paper is desired,
% comment out the letterpaper lines and uncomment the a4paper lines.
%
% As you can see, the margin widths and section title widths can be
% easily adjusted.
%
% ALSO: Notice that the includefoot option can be commented OUT in order
% to put the PAGE NUMBER *IN* the bottom margin. This will make the
% effective text area larger.
%
% IF YOU WISH TO REMOVE THE ``of LASTPAGE'' next to each page number,
% see the note about the +LP and -LP lines below. Comment out the +LP
% and uncomment the -LP.
%
% IF YOU WISH TO REMOVE PAGE NUMBERS, be sure that the includefoot line
% is uncommented and ALSO uncomment the \pagestyle{empty} a few lines
% below.
%

%% Use these lines for letter-sized paper
\usepackage[paper=letterpaper,
            %includefoot, % Uncomment to put page number above margin
            marginparwidth=1.2in,     % Length of section titles
            marginparsep=.05in,       % Space between titles and text
            margin=1in,               % 1 inch margins
            includemp]{geometry}

%% Use these lines for A4-sized paper
%\usepackage[paper=a4paper,
%            %includefoot, % Uncomment to put page number above margin
%            marginparwidth=30.5mm,    % Length of section titles
%            marginparsep=1.5mm,       % Space between titles and text
%            margin=25mm,              % 25mm margins
%            includemp]{geometry}

%% More layout: Get rid of indenting throughout entire document
\setlength{\parindent}{0in}

%% This gives us fun enumeration environments. compactitem will be nice.
\usepackage{paralist}

%% Reference the last page in the page number
%
% NOTE: comment the +LP line and uncomment the -LP line to have page
%       numbers without the ``of ##'' last page reference)
%
% NOTE: uncomment the \pagestyle{empty} line to get rid of all page
%       numbers (make sure includefoot is commented out above)
%
\usepackage{fancyhdr}
\usepackage{lastpage}
\pagestyle{fancy}
%\pagestyle{empty}      % Uncomment this to get rid of page numbers
\fancyhf{}\renewcommand{\headrulewidth}{0pt}
\fancyfootoffset{\marginparsep+\marginparwidth}
\newlength{\footpageshift}
\setlength{\footpageshift}
          {0.5\textwidth+0.5\marginparsep+0.5\marginparwidth-2in}
\lfoot{\hspace{\footpageshift}%
       \parbox{4in}{\, \hfill %
%                     \arabic{page} of \protect\pageref*{LastPage} % +LP
                   \arabic{page}                               % -LP
                    \hfill \,}}

% Finally, give us PDF bookmarks
\usepackage{color,hyperref}
\definecolor{darkblue}{rgb}{0.0,0.0,0.3}
\hypersetup{colorlinks=false,breaklinks,
            linkcolor=darkblue,urlcolor=darkblue,
            anchorcolor=darkblue,citecolor=darkblue}

%%%%%%%%%%%%%%%%%%%%%%%% End Document Setup %%%%%%%%%%%%%%%%%%%%%%%%%%%%


%%%%%%%%%%%%%%%%%%%%%%%%%%% Helper Commands %%%%%%%%%%%%%%%%%%%%%%%%%%%%

% The title (name) with a horizontal rule under it
%
% Usage: \makeheading{name}
%
% Place at top of document. It should be the first thing.
\newcommand{\makeheading}[1]%
        {\hspace*{-\marginparsep minus \marginparwidth}%
         \begin{minipage}[t]{\textwidth+\marginparwidth+\marginparsep}%
                {\large \bfseries #1}\\[-0.15\baselineskip]%
                 \rule{\columnwidth}{1pt}%
         \end{minipage}}

% The section headings
%
% Usage: \section{section name}
%
% Follow this section IMMEDIATELY with the first line of the section
% text. Do not put whitespace in between. That is, do this:
%
%       \section{My Information}
%       Here is my information.
%
% and NOT this:
%
%       \section{My Information}
%
%       Here is my information.
%
% Otherwise the top of the section header will not line up with the top
% of the section. Of course, using a single comment character (%) on
% empty lines allows for the function of the first example with the
% readability of the second example.
\renewcommand{\section}[2]%
        {\pagebreak[2]\vspace{1.3\baselineskip}%
         \phantomsection\addcontentsline{toc}{section}{#1}%
         \hspace{0in}%
         \marginpar{
         \raggedright \scshape #1}#2}

% An itemize-style list with lots of space between items
\newenvironment{outerlist}[1][\enskip\textbullet]%
        {\begin{itemize}[#1]}{\end{itemize}%
         \vspace{-.6\baselineskip}}

% An environment IDENTICAL to outerlist that has better pre-list spacing
% when used as the first thing in a \section
\newenvironment{lonelist}[1][\enskip\textbullet]%
        {\vspace{-\baselineskip}\begin{list}{#1}{%
        \setlength{\partopsep}{0pt}%
        \setlength{\topsep}{0pt}}}
        {\end{list}\vspace{-.6\baselineskip}}

% An itemize-style list with little space between items
\newenvironment{innerlist}[1][\enskip\textbullet]%
        {\begin{compactitem}[#1]}{\end{compactitem}}

% An environment IDENTICAL to innerlist that has better pre-list spacing
% when used as the first thing in a \section
\newenvironment{loneinnerlist}[1][\enskip\textbullet]%
        {\vspace{-\baselineskip}\begin{compactitem}[#1]}
        {\end{compactitem}\vspace{-.6\baselineskip}}

% To add some paragraph space between lines.
% This also tells LaTeX to preferably break a page on one of these gaps
% if there is a needed pagebreak nearby.
\newcommand{\blankline}{\quad\pagebreak[2]}

% Uses hyperref to link DOI
\newcommand\doilink[1]{\href{http://dx.doi.org/#1}{#1}}
\newcommand\doi[1]{doi:\doilink{#1}}


%%%%%%%%%%%%%%%%%%%%%%%% End Helper Commands %%%%%%%%%%%%%%%%%%%%%%%%%%%

%%%%%%%%%%%%%%%%%%%%%%%%% Begin CV Document %%%%%%%%%%%%%%%%%%%%%%%%%%%%

\begin{document}
\makeheading{Maxym V. Myroshnychenko}

\section{Contact Information}

\newlength{\rcollength}\setlength{\rcollength}{1.85in}
\begin{tabular}[t]{@{}p{\textwidth-\rcollength}p{\rcollength}}
 & \\
\vspace{-.45in}
    \textit{E-mail:}\linebreak
\href{mailto:mmyros@gmail.com}{mmyros@gmail.com}\\
%\vspace{.1in}
 {\it Web:}\\
 \href{mmyros.github.io/github.io/}{mmyros.github.io/github.io/}\\
\end{tabular}

\section{Education}
\textbf{Indiana University}
\begin{innerlist}
\item[] Ph.D. candidate,
             Program in Neuroscience \hfill {August 2011 to present}
\end{innerlist}

\textbf{University of Nevada, Las Vegas}
\begin{innerlist}
	\item[] B.S.,
             Biology, Biomathematics \hfill {August 2011}

\end{innerlist}

\section{Laboratory affiliations}
\href{http://science.iupui.edu/people/christopher-lapish}{\textbf{Lapish laboratory}}
\begin{outerlist}
\item[] \textit{PI: Christopher Lapish}
  \hfill{August 2014 to present}
    \begin{innerlist}
    \item Tetrode and shank recordings from anesthetized and awake rats
    \item Data analysis: state space, spectral decomposition, machine learning
    \item Development of automatic behavior collection and analysis tools
    \item Contribution to VTA/mPFC computational modeling (collaboration with A. Kuznetsov, IUPUI Math)
    \item TH-Cre optogenetics with anesthetized electrophysiology
    \end{innerlist}
\end{outerlist}
\vspace{.15in}

\href{http://www.indiana.edu/~iubphys/faculty/jmbeggs.shtml}{\textbf{Beggs neuronal dynamics laboratory}}
\begin{outerlist}
\item[] \textit{PI: John Beggs}
  \hfill{August 2011 to to August 2014}
    \begin{innerlist}
    \item Statistical analyses of in vitro 512-electrode array data from mouse/rat hippocampus (collaboration with Litke lab, UC Santa Cruz),in vivo Utah array primate data (collaboration with Hatsopulous lab, U. of Chicago)
    \item Transfer entropy, mutual information, community detection.
    \end{innerlist}
\end{outerlist}

\section{Research projects}
\textbf{Cortico-hippocampal interactions in the radial arm maze}
\begin{innerlist}
\item[] Characterizing the process of planning in delayed spatial win-shift task using optogenetic inhibition of HC; shank and wire mPFC recordings 
\end{innerlist}
\blankline

\textbf{Effects of ethanol on interactions beetween VTA GABA and DA neurons and mPFC}
\begin{innerlist}
\item[] Dissecting local and distal dynamic connectivity of ventral tegmental area using dual-site single-unit recordings, optogenetic stimulation, and pharmacological manipulations as a part of France-USA computational modeling collaboration
\end{innerlist}
%%%%%%%%%%%%%



\bibliographystyle{plain}
\section{Publications} \begin{bibsection}

\item Myroshnychenko M,  Lapish CC. Contributions of hippocampal input to dynamics of medial prefrontal cortex during and after delay of a spatial working memory task. In preparation
  
\item Myroshnychenko M,  Seamans JK, Philips AG,  Lapish CC. Temporal dynamics of hippocampal and medial prefrontal cortex interactions during the delay period of a spatial working memory task. Cerebral cortex, in print

\item Morozova E O, Myroshnychenko M, Zakharov D, di Volo M, Gutkin B, Lapish C, Kuznetsov A (2016). Contribution of synchronized GABAergic neurons to dopaminergic neuron firing and bursting. Journal of Neurophysiology, 116(4), 1900-1923.

\item Timme NM, Ito S, Myroshnychenko M, Nigam S, Shimono M, Yeh FC, Hottowy P, Litke AM, Beggs JM. (2016) High-Degree Neurons Feed Cortical Computations. PLoS Comput Biol. May 9;12(5):e1004858.

\item Nigam S, Shimono M,  Ito S,  Yeh F, Timme N,  Myroshnychenko M,  Lapish C,  Tosi Z,  Hottowy P,  Smith W,  Masmanidis S,  Litke A,  Sporns O,   Beggs JM. (2016) Rich-club organization in the functional micro-connectome. \emph{Jounral in Neuroscience} Jan 20;36(3):670-84. 

\item Timme N, Ito S, Myroshnychenko M, Yeh F, Hiolski E, Hottowy P, Beggs JM. (2014) Multiplex networks of cortical and hippocampal neurons revealed at different timescales. \emph{PLoS ONE}  9(12): e115764.. 



\end{bibsection}

\section{Presentations} \begin{bibsection}

\item Myroshnychenko M,  Lapish CC. Prefrontal-hippocampal theta coherence, sharp wave ripples, and bursts of cortical unit activity underlie choices and encoding in the radial arm maze. Poster presentation, Society for Neuroscience meeting, Chicago, IL, 2015


\item Myroshnychenko M,  Lapish CC. Prefrontal-hippocampal theta coherence, sharp wave ripples, and bursts of cortical unit activity underlie choices and encoding in the radial arm maze. Poster presentation, Society for Computational Neuroscience meeting, Prague, Czech Republic, 2015


\item Myroshnychenko M, Morozova E, Kuznetsov A, Lapish CC. Dissecting reward circuitry with simultaneous single-unit recording in PFC and VTA. Poster presentation, Research society for alcohol, San Antonio, TX, 2015

\item Myroshnychenko M, Morozova EO,  Kuznetsov A, Lapish CC. Dissecting reward circuitry with simultaneous single-unit recording in PFC and VTA. Poster presentation, Indianapolis chapter of Society for Neuroscience meeting, 2014

\item Myroshnychenko M,  Nicholson B,  Yeh F,  Brickman B,  Dahmen K,  Litke A,  Beggs J. Critical features of massively parallel cortical single-unit recordings. Poster presentation, Gill symposium, Indiana University, 2013


\item Sarine S. Janetsian, Maxym Myroshnychenko, Christopher C. Lapish. Changes in neuronal firing and oscillatory activity in the PFC following Methamphetamine sensitization. Poster presentation, Society for Neuroscience meeting, 2013

\item Myroshnychenko MV, Heaney CF, Bolton MM, Sabbagh JJ, Kinney JW  "Acute Administration of Ketamine Impairs Learning in
	Trace Cued Fear Conditioning: Validation of an Animal Model
	of Schizophrenia."  21th Annual McNair Research Conference. Oklahoma State University. February 24, 2011

\item Myroshnychenko MV, Heaney CF, Bolton MM, Sabbagh JJ, Kinney JW. "Acute Administration of Ketamine Impairs Learning in
	Trace Cued Fear Conditioning: Validation of an Animal Model
	of Schizophrenia."The 2010 McNair Scholars Institute poster presentation. University of Nevada, Las Vegas, NV. October 21, 2010.
	
\item Myroshnychenko MV, Estevez J, Harbour D.  "{\it Krameria erecta} and {\it Oenotheria biennis} extracts increase density of {\it Staphylococcus epidermidis} biofilm."  The 2010 McNair Scholars Institute poster presentation. University of Nevada, Las Vegas, NV. October 21, 2010.
    \item Zarrabi K, Nitrosesatien N, Koh J, Naserddin S, Abanyan E, Myroshnychenko M, 
Esteves J, Harbour D, Porter H. Antibacterial Potential and GC-MS Studies of Select 
Medicinal Plants of Mojave Desert. Presented at the 2009 Northwest Regional Meeting 
of the American Chemical Society, Pacific Lutheran University, Tacoma, WA.
\end{bibsection}

\pagebreak

\section{Skills}
%\vspace{-.15in}
Experimental techniques
\begin{innerlist}
\item Stereotaxic surgery
\item  Awake behaving/anesthetized extracellular electrophysiology (tetrodes, Neuronexus, Harris, Masmanidis shanks, gold and PEDOT coating)
\item Spikesorting shank data (spyking circus, phy packages)
\item Optogenetics (programming PulsePal driver)
\end{innerlist}

%\vspace{.15in}
Programming
\begin{innerlist}
\item Matlab,  Python, mex/C, github, Linux, clusters
\end{innerlist}

%\vspace{.15in}
Real-time processing
\begin{innerlist}
\item Linux OS, Open Ephys, Arduino, simple GUIs, computer vision
\end{innerlist}


\section{Summer school attendance}
 CoSMo \hfill {June 2013}	
\begin{innerlist}
\item  Computational Sensory-Motor Neuroscience, organizer K. Kording
\item  Machine learning, Bayesian and neural net approaches to decoding
\end{innerlist}
\vspace{.15in}
 CRCNS \hfill {July 2014}	
\begin{innerlist}
\item  Berkeley summer course in mining and modeling of neuroscience data, organizers Jeff Teeters and Fritz Sommer
\item  STC, model fitting, ICA, GLM
\end{innerlist}

\blankline

\section{Awards}
Fellowships
\begin{innerlist}
\item National Science Foundation Biomathematics
			Scholar \hfill {May 2010 - May 2011}	
\item University of Nevada, Las Vegas McNair Summer Institute Fellowship \hfill{May 2010}
\end{innerlist}
\blankline

Scholarships and grants
\begin{innerlist}
\item University of Nevada, Las Vegas Scholarship \hfill{November 2009}
\item College of Southern Nevada Scholarship \hfill{November 2008}
\item Federal {\sc smart} grant \hfill{2009 - 2011}
\end{innerlist}

\blankline

\section{Teaching Experience}

{\textbf{Indiana University}}, 
\begin{outerlist}
\item[] \textit{Teaching assistant}%
    \hfill {Fall 2014}

Addiction neuroscience lecture and lab. Responsible for grading, lab preparation
\end{outerlist}
\vspace{.15in}

{\textbf{The Lovaas Center of Las Vegas}}, Las Vegas, Nevada
\begin{outerlist}
\item[] \textit{Tutor}%
    \hfill {June 2009 to August 2009}

         Applied Behavioral Analysis for children with autism.
\end{outerlist}
\vspace{.15in}
{\textbf{College of Southern Nevada}},
Las Vegas, Nevada
\begin{outerlist}
\item[] \textit{Tutor}%
    \hfill {September 2009 to May 2009}

         Responsible for coaching students on various subjects including biology, writing, and mathematics. 

\end{outerlist}

\section{References}
\vspace{-.25in}
\begin{outerlist}
\item Dr. Christopher Lapish \href{mailto:lapisc@gmail.com}{lapishc@gmail.com}, Indiana University Purdue University Indianapolis
\item Dr. Alexey Kuznetsov  \href{mailto:askuznet@gmail.com}{askuznet@gmail.com}, Indiana University Purdue University Indianapolis
\item Dr. John Beggs \href{mailto:jmbeggs@indiana.edu}{jmbeggs@indiana.edu}, Indiana University 
\end{outerlist}

\end{document}


%%%%%%%%%%%%%%%%%%%
\section{Professional Society Affiliations}
%
Member of:
\begin{innerlist}
\item Computational Neuroscience society
\item Research society on Alcoholism
\item Society for Neuroscience
\item Molecular and Cellular Cognition society 
%\item The Bioinformatics Organization
%\item \href{http://www.ptk.org/}{Phi Theta Kappa} Honors Society of two-year colleges 
\end{innerlist}






%%%%%%%%%%%%%%%%%%%%%%%
\section{Research projects}
Indiana University
\begin{outerlist}
\item C

\end{outerlist}

University of Nevada, Las Vegas
\begin{outerlist}
	\item Compounds affecting learning and memory in a rat model of schizophrenia. Includes behavioral assessment in Morris water maze and fear conditioning, surgery, and tissue work. 
Principal investigator: Professor Jefferson Kinney, University of Nevada, Las Vegas \hfill{May 2010-January 2011}
\end{outerlist}

\blankline

College of Southern Nevada
\begin{outerlist}	
\item Effects of select medicinal plants of Mojave desert on biofilm formation by common 
nosocomial pathogens. Principal investigator: Dr. D. Harbour, College of Southern 
Nevada \hfill{June-August,2009}
	\item Antibacterial effects of select medicinal plants of Mojave Desert. Principal investigator: 
Dr. K. Zarrabi, College of Southern Nevada  \hfill{January-May,2009}
\end{outerlist}

\section{Research Appointments}
%
\textbf{Graduate Research Assistant} \hfill {May 2010 to present}
\begin{innerlist}
\item[] {Department of Physics},\\
        Indiana University
        \href{http://www.indiana.edu/~iubphys/faculty/jmbeggs.shtml}{John Beggs' Laboratory}\\
        Principal investigator: Associate Professor John Beggs
%         ---------------------------------------------

    \end{innerlist}
    \textbf{Research Assistant} \hfill {May 2010 to January 2011}
\begin{innerlist}

\item[] \href{http://www.psychology.unlv.edu/}{Department of Psychology},\\
        University of Nevada, Las Vegas
        \href{http://faculty.unlv.edu/jkinney/}{Behavior Neuroscience Laboratory}\\
        Principal investigator: Professor Jefferson Kinney

\end{innerlist}


Others
\begin{innerlist} 
\item Recruitment grant. University of Rochester \hfill{September 2010}
\item Recruitment grant. University of Indiana, \newline Bloomington \hfill{September 2010}
\item BIOCOMP Conference grant. Biomathematics Scholars program \hfill{June 2010}
\item Dean's Honors list. University of Nevada, Las Vegas \hfill{May 2010}

\end{innerlist}



% \pagebreak
\section{Research 
Skills}
Behavioral techniques: 
\begin{innerlist}
\item Morris water maze 
\item Cued and contextual fear conditioning
\item Tail flick test
\item Prepulse inhibition test
\end{innerlist}

\blankline

Bench techniques:
\begin{innerlist}
\item Western blotting
\item Surgery, including perfusion and dissection
\item Immunohistochemistry using the ABC kit
\item Mass spectrometry
\item Affinity chromatography
\item Gas chromatography 
\end{innerlist}

\blankline

Computer Applications: 
\begin{innerlist}
\item R 
\item Mathematica 
\item {\sc matlab}
\item Linux operating system
\item SPSS
\item \TeX{} (\LaTeX{}, B\textsc{ib}\TeX{})
\end{innerlist}

\section{Coursework}
% \blankline
\textbf{University of Nevada, Las Vegas}
\begin{outerlist}
\item[] \textit{Advanced Independent Study}    \hfill {Spring - summer 2011}	 
	\begin{innerlist}
		\item A study of Bayesian modeling		
% 		\item A statistics course developed in collaboration with a faculty member from the department of Mathematics

% 		\item Self-developed curriculum will include independent readings and discussion with the faculty mentor
	\end{innerlist}
%\end{outerlist}
\item[] \textit{Biomathematics}    \hfill{Fall 2010, Spring 2011}
	\begin{innerlist}
		\item An experimental, NSF-funded two-semester course with a total enrollment of four students
		\item Discrete difference equations, differential equations, probability theory, which are applied to mathematical models of biological systems
		\item Software applications including Mathematica, {\sc matlab} and Maple
		\item Second semester of the course includes design and completion of an independent project
	\end{innerlist}
\item[] \textit{Current topics in statistics and applied mathematics} \hfill{February 26, 2010}
	\begin{innerlist}
		\item Applied for funding
		\item Undergraduate workshop funded by the Statistical and Applied Mathematical Sciences Institute
		\item Topics include analysis of object data among others
	\end{innerlist}
\end{outerlist}	
	\begin{outerlist}
\item[] \textit{Advanced Topics in Molecular Biology}    \hfill {Fall 2010}	 
	\begin{innerlist}
		\item Graduate-level course based on primary scholarship
		\item Weekly discussion of latest articles from \textit{Cell, Nature}, and \textit	{Science}
		\item Students present recent publications of their choice
	\end{innerlist}
\end{outerlist}

\blankline

\textbf{College of Southern Nevada}
\begin{outerlist}
\item[] \textit{Honors Microbiology}    
 \hfill{Spring 2009}
	\begin{innerlist}
		\item A primary literature intensive course
		\item A total enrollment of six students 
	\end{innerlist}
	
For a description of additional coursework, please refer to the attached unofficial transcipt. 
\end{outerlist}

\end{document}

%%%%%%%%%%%%%%%%%%%%%%%%%% End CV Document %%%%%%%%%%%%%%%%%%%%%%%%%%%%%
